
\chapter{Introduction \& Overview}

\section{Introduction To Project}

For my final year project I have created a compiler in Haskell for a simple programming language of my own design, which I based on Google's relatively new language Go. I called this language Goal, and it is a simplified version of Go but contains all of the features I was interested in implementing in my compiler, most importantly allowing concurrent programming.

The main focus of this project was to implement a parser, compiler and executor for a modern concurrent programming language, not on designing a completely new language. Though I will briefly discuss the reasons behind creating Goal, it is important to understand the focus of this project is on compiler implementation not programming language design.

This document is split into five main sections; Designing Goal, Parsing, Code Generation \& Intermediate Representations, Code Execution and Testing. In each of these sections I will go into more detail about how I approached each of these problems. 

These next few pages will give a brief introduction to compilers and programming languages, also giving more information about Go and the motivation behind the project. I will also briefly discuss any development methodologies used and give a justification for certain technical decisions.

\subsection{Introduction to Compilers \& Executors}

To understand what a compiler is, you need to first understand what a program is and, more importantly, what a programming language is. Programs have become a fundamental part of human existence. From performing simple calculations to creating applications that allow me to create documents such as this, they exist in every aspect of our life. The simplest definition of what a program is, is to think of a program as a recipe.  

You have a list of resources that you need, followed by a series of instructions telling you what to do with these resources. Then, to run your program, you grab all the resources you need and follow the instructions.

Programs that run on computers have to be represented in someway that both humans and eventually computers can understand.  Therefore programs are written in programming languages, a good definition for programming languages can be found in  \cite[p,~1]{Comp2007}  where they state;

\begin{displayquote}
Programming Languages are notations for describing computations to people and machines … But, before a program (in this format) can be run it first must be translated into a form which can be executed by a computer.
\end{displayquote}

So this brings us nicely to being able to define what a compiler is, a compiler is something that takes programs written in one programming language and translates it into another programming language. Often taking a program written in a high level language, that humans write code in, and translating them into lower level languages that computers can more easily understand and then execute.

Finally an executor is something that executes code written in a given programming language. You will usually only see executors created to execute low level languages given the smaller, and more precisely defined, instruction set. As is the case with the executor I have created. This highlights the need for compilers, so that they can be used to translate high level languages into more easily executable lower level languages, that are much harder for humans to create programs in. 

Looking at these explanations it is clear to see the importance of compilers in the modern world. They allow people to create complicated programs and applications efficiently without having to concern themselves with low level details.

\subsection*{Introduction to Compiler Structure}



\subsection{Introduction to Goal}

Goal is the language I have created my compiler for. It is a language I have created specifically for this project as a means of exploring compilers. It draws heavy inspiration from Go and uses the same basic syntax where possible. The main features of Goal are the ability to perform function recursion and to create and run concurrent programs.

The idea for Goal came when i was picking features from Go I was interested in compiling then decided it would be a nice idea to bring them together to create a more complete language. In my opinion during the course of this project Goal has evolved from a means to explore compilers and language design into a simple standalone language with it's own uses. I will discuss more about the design and functionality of Goal in chapter 2.


\subsection{Introduction to Go}

Go is an object oriented programming language created by Google relatively recently in 2007. It is used by Google for many different applications, most notably it powers dl.google.com, a service which contains the source for Chrome and The Android SDK. A good place to find out more about the history of Go and it's development is in the documentation section on golang.org.

In many ways Go is similar to C in terms of syntax and that it is also statically typed. There are however some interesting features thrown in, such as how Go handles concurrency and even some newer ideas such as splices, which we will talk about later.

In summary I would say Go (also referred to as golang) is a new language with quite a bit of potential, which can be seen by it's growing popularity. Although it is not revolutionary, it's mixture of simplistic syntax and more exciting features makes it a good language to emulate.

\subsection{Introduction to Concurrency}

Concurrency is the primary feature I was interested in implementing in my compiler. It is a key part of this project and also a fundamental part of modern programming. 

The basic idea behind concurrent programming is you can do many things at one time. If we first define a sequential program as a sequence of operations carried out one at a time. We can then define a concurrent program as a program that contains a set of sequential processes executing in parallel \cite[p.~414]{CompGen1997}.

The importance of concurrent programming can be seen now more than ever with rising popularity of online services and cloud computing. Lets take one of the most popular websites in the world Google.com, with around 40,000 request a second there is a high chance that when you hit that search button, so are thousands of other people at exactly the same time. But instead of queuing every request and doing it sequentially a server will handle the requests concurrently, ensuring that many requests can be handled at the same time, and you don't have to wait for the thousands of people who clicked a few milliseconds ahead of you till you get your results.

There are plenty more examples of the importance of concurrency in modern technology, I'm sure if you think about most pieces of technology you use it would be easy to list numerous process that have to run concurrently. This is why I was interested in implementing concurrency into my compiler, because of it's usefulness and importance in modern programming languages.   


\section{Motivation}

The biggest motivation behind this project was a desire to learn more about compiling and executing modern object oriented programming languages. I was also curious about programming language design and wanted to take the chance to really look at a programming language analytically.

\subsection{Why Haskell}

I chose to implement my compiler using Haskell. Haskell is a purely functional language with strong static typing. Functional languages are often seen as good platforms to use when creating a compiler because some of their features make it easier to handle tree data structures, which can be important during parsing and creating an intermediate representation of language. Also the use of pattern matching and efficiently being able to recurse makes Haskell a good choice for implementing a compiler and a virtual machine.

Another reason I chose Haskell was quite simply that it is a language I enjoy working with, and more importantly for a project like this, would like to learn more about. I was keen with this project to not only create something interesting but also learn more about functional programming. I hoped to use this project to explore some  unique functional approaches to some of the problems creating a compiler can throw up. A good example of this is how I handled parsing and my use of Monadic Parser Combinators.

\subsection{Why Create A New Language}

There are so many great programming languages out there with fantastic supporting documentation that would be excellent choices to make compilers for. Which really begs the question, why did I go through the hassle of designing my own language to compile? The answer to this is pretty simple, with a project like this if I was to create a compiler for C++, for example, I would never have the time to create something that covers all of C++'s functionality within 9 months. Therefore no matter how well I had done with the project it would always feel a little incomplete, but more importantly if someone was to use my compiler they could very well end up trying to use features my compiler didn't support which would be frustrating for me and anyone who wishes to use my compiler in the future.

Instead of having this problem I decided I would create a new language that I could provide supporting documentation for and strictly define what is and isn't possible with in it. Because the focus of this project was not on language design my approach to creating a new programming language was finding a language I liked, then picking key features I was interested in exploring. Before finally adding some extra functionality so that this new language could be useful on its own.

The language I liked the look of was Go and that is where most of the features for my language came from, with the key feature I was interested in exploring being concurrency. I decided to call the language I had created Goal.

\subsection{Why Base This Language on Go} 

Go is not a language I was overly familiar with before the start of this project, but as soon as I looked into it I very much liked what I saw. 

At the start of this project I was exploring different possibilities of languages I could use as inspiration for Goal. When I came across Go I discovered it had a very clean and easy to understand way of creating and running concurrent process, which was one of the key features I was interested in putting into Goal. From there I began looking into Go's design and discovered not only did it have a wealth of interesting features to look at but also a clean and understandable syntax, which would work nicely with the simplistic easy to use nature I wanted to create in Goal.

The most important factors where defiantly the way Go handle concurrency and channels but also there was an aspect of using this project as a way of familiarizing myself with another language. It also helped that Go came with such a wealth of easy to navigate online resources which was perfect for when I needed to check the exact functionality of certain expressions.

Overall I felt Go provided me with a simple syntax and good implementations of the main features I was interested in implementing. Which made it a good choice to use as the basis of Goal.

\section{Development Process }

The main method I used for development was Test Driven Development (TDD). This is a simple approach where you write your test cases first, then write your code so that is passes all your tests. I found this to be quite an appropriate approach due to the iterative nature of my development process. 

I initially started work by focusing on being able to compile and execute a small subset of simple features such as if statements, variable assignments and simple arithmetic. Then using TDD I wrote tests for each new piece of functionality I wanted to add, expanding the subset of features I was able to handle. Due to the nature of TDD I was able to do this without worrying about breaking earlier features as I could just ensure all my original tests still passed.

As was mentioned before, I split the implementations of my compiler into three main sections;

\begin{itemize}
	\item Parsing
	\item Code Generation \& Intermediate Representations 
	\item Execution, using a Stack Based Virtual Machine
\end{itemize}

While developing my compiler and executer I often found I would implement each feature in two steps; First ensuring I could create a suitable intermediate representation and updating my virtual machine to ensure it could handle the new feature. Then, once this was complete, I would update my parser to handle this new command. 

Often when working on multiple features at a time, it would be more efficient to update the code generation and virtual machine first for all the new features. Then update the parser second, rather than adding each new piece of functionality one at a time. 

I will go into more detail about the creation and running of my tests in Chapter 5. 
   