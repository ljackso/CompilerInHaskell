
\chapter{Designing Goal}

This chapter deals with how I approached designing a new language and how I went about choosing the features I wished for Goal to contain. I will discuss some reasoning behind why certain features differ from how they are handled in Go and also talk about how I tried to make the language as complete as possible.

I think it is also important to note here that the main focus of this project is not on the design and creation of a new language but rather the implementation of the compiler itself.

\section{Picking Features}

As I mentioned before, rather than designing Goal from scratch I chose to instead choose features from Go I was interested in, then find a way of putting them into Goal. I did not create Goal with a target audience or with potential uses in mind, hence the lack of a specification or market research. Instead I created Goal as a means to let me implement a compiler that could handle a number of different interesting features.

Therefore you will find the features Goal can handle my seem quite varied and not completely complimentary of each other, but I do feel that Goal still has many uses. 

As much as Goal was created simply as a tool for creating an interesting compiler, I did also attempt to make it as user friendly as possible. I have created supporting documentation for Goal that has full examples of the syntax it uses and detailed explanations of how to use each of it's features.  
 
A good summary of my approach to creating Goal is that I filled it with features I wanted to explore, then added extra functionality to try and make the language as easy to use, complete and useful as possible.
 
\subsection{Syntax}

A language's syntax can be said to describe the form that commands and expressions in a language must take \cite[p.~72]{CompGen1997}. In the case of programming languages it defines how you must write your code so that it can perform the computations you wish.

The syntax for Goal is pretty simple and almost identical to that of Go, with some minor differences. I decided to follow Go's syntax rules not only for simplicity but also so it was easy to see where Goal got it's functionality from. I felt it would make sense to give it syntax rules close to the language it was emulating.
 
A key part of Goal's syntax is that each command must end with a semi-colon, including if statements, functions and for loops. A more detailed outline and examples of valid Goal syntax can be seen in the accompanying documentation for Goal with examples of syntactically correct Goal \cite{GoalDoc}. 

\subsection{Types}

\subsection{Basic Commands}

\subsection{Functions}

\subsection{Concurrency}



\section{Differences From Go}

\section{Possible Uses}
