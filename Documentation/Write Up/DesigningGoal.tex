\chapter{Designing Goal}

This chapter deals with how I approached designing a new language and how I went about choosing the features I wished for Goal to contain. I will discuss some reasoning behind why certain features differ from how they are handled in Go and also talk about how I tried to make the language as complete as possible.

I think it is also important to note here that the main focus of this project is not on the design and creation of a new language but rather the implementation of the compiler itself.

\section{Picking Features}

As I mentioned before, rather than designing Goal from scratch I chose to instead choose features from Go I was interested in, then find a way of putting them into Goal. I did not create Goal with a target audience or with potential uses in mind, hence the lack of a specification or market research. Instead I created Goal as a means to let me implement a compiler that could handle a number of different interesting features.

Therefore you will find the features Goal can handle my seem quite varied and not completely complimentary of each other, but I do feel that Goal still has many uses. 

As much as Goal was created simply as a tool for creating an interesting compiler, I did also attempt to make it as user friendly as possible. I have created supporting documentation for Goal that has full examples of the syntax it uses and detailed explanations of how to use each of it's features.  
 
A good summary of my approach to creating Goal is that I filled it with features I wanted to explore, then added extra functionality to try and make the language as easy to use, complete and useful as possible.
 
\subsection{Syntax}

A language's syntax can be said to describe the form that commands and expressions in a language must take \cite[p.~72]{CompGen1997}. In the case of programming languages it defines how you must write your code so that it can perform the computations you wish.

The syntax for Goal is pretty simple and almost identical to that of Go, with some minor differences. I decided to follow Go's syntax rules not only for simplicity but also so it was easy to see where Goal got it's functionality from. I felt it would make sense to give it syntax rules close to the language it was emulating.
 
A key part of Goal's syntax is that each command must end with a semi-colon, including if statements, functions and for loops. A more detailed outline and examples of valid Goal syntax can be seen in the accompanying documentation for Goal with examples of syntactically correct Goal \cite{GoalDoc}. 

\subsection{Types and Scope}

A type system defines what type of variables are allowed to be used in a language. A variables type can be defined by saying \cite[p.~164]{EngComp2012};

\begin{displayquote}
The type (of a variable) specifies a set of properties held in common by all values of that type ... For example, an integer might be any whole number $i$ in the range $-2^{31} \leq i < 2^{31}$.
\end{displayquote}

Go has a very broad type system and static typing. Static typing is where any type errors are checked during compiling as opposed to dynamic typing where type errors are checked at run time. By saying Go has a broad type system I'm saying it allows for a lot of different types.

In Goal I decided not to focus on allowing lots of different types and instead only allow 3 main types; Integers,  Booleans and Strings. This was mainly so I could spend more time working on other features and not worrying about creating an extensive type checking system, whilst still having enough types to be able to create some interesting programs. 

This also lead to allowing very basic variable declarations and assignments. You do not need to declare a type when initially declaring a variable and variables can be created on the fly. For example if you were to write;

\begin{lstlisting}
	i = 4;
	i = True;
	i = "hello";
\end{lstlisting}

There would be no compiler or run time errors in this code and your variable $i$ would have the value "hello" as each new assignment writes over the old value. This is the biggest difference Goal and Go, it is actually much closer to how a language like Python handles variable declarations and puts more emphasis on the user keeping track of their variables.

A variable's scope defines where it can be accessed from. In Goal I decided to keep this very simple by only allowing variables to have two different scopes; global or local. A global variable is defined by using the keyword global before the definition and once declared can be accessed anywhere in your code. A local variable is just declared as shown above with no key word and can only be accessed with in the function you declared it in. More information about how variable scope works in Goal can be found in the accompanying documentation.     


\subsection{Basic Commands}

By basic commands I mean the set of statements you expect to find in most object orientated languages. In Goal this includes;

\begin{itemize}
	\item If Statements
	\item While Loops
	\item For Loops
	\item Output Commands
	\item Return Statements 
\end{itemize}

All of these almost exactly follow the same syntax rules and have the same functionality that you find in Go. There are some small differences such as in Goal you are required to hold your conditional expression in brackets in each command but other than that there is not much difference. Again for more clarification on how these statements work you can find examples in the Goal documentation.

\subsection{Functions}

Functions are treated very similarly in Goal as they are in Go. The main difference is that you can only have functions that either return Booleans or Integers, or your functions can be void. Function recursion is allowed as are nested function calls.

\subsection{Concurrency}

For concurrency I wanted to use the same simple technique that allowed you to run any void function as a concurrent process with the use of a keyword. Meaning if you wanted to run two functions $funA()$ and $funB()$ concurrently all you needed to do was write;

\begin{lstlisting}
	go funA();
	go funB();	
\end{lstlisting}
This code would start both the processes as independent subroutines and would execute them in parallel. I also added some useful library functions that I felt would be helpful for creating concurrent programs. These are not found in Go but are important to the way Goal works. The two most important ones are $Wait()$, which will halt a program until all subroutines have finished executing, and $Kill()$ which will immediately stop all running subroutines. There are a few other library functions I have included and again more information can be found in Goal's Documentation.  

I also needed to implement a way to pass information between subroutines, although you could use global variables it is safer to have a more controlled method. This is why I implemented channels. These are also present in Go. You can declare a channel anywhere and it has global scope. They behave like stacks so you can push a value onto a channel in one subroutine then pop it out in another subroutine. Channels have a first in first out system. I have implemented channels almost exactly the same as how they are implemented in Go. 

\section{Differences From Go}

Although the language I have created is based on Go there are quite a few noticeable differences. The biggest difference is that Go is statically typed and Goal I have made dynamically typed, although there is a very basic type checker in place at compile time. This mainly means that if you convert your Go code to Goal you need to take care that you don't overwrite any variables.

Other noticeable differences come from the output functions being part of Goals standard library where as in Go you eed to import a package to output to a console.

Obviously the scope of Goal is much smaller as it has a much more limited type system and does not allow for multiple classes or files. Though it is noticeably similar and does contain many of the key features that Go contains 

\section{Possible Uses}

Towards the end of this project I began to reflect on Goal as a standalone language and not just as something for me and felt it could have some uses of it's own.

Where Go is most popular at the moment is for a 

Although I don't think Goal has a future competing with Go as tool for implementing large scale servers. I do feel that given Goal's simplistic syntax and easily understandable ,and usable, features it could have some use as a tool for teaching programming. More specifically helping people to understand how concurrent programs work, and giving them the tools to create their own exciting concurrent programs.

You can see I have provided with my project some example programs and as part of that I have included some programs I feel highlight how Goal could be used as a teaching tool in the future. I will revisit this idea as part of my evaluation of the project as whole.    
