
\chapter{Code Generation \& Intermediate Representation}

Code generation is where I take the parsed input and then generate code from a low level instruction set that can then be run by my virtual machine. I decided to implement my own low level language and create a virtual machine to run code written in that language. Thus meaning that I would be compiling the parsed input down to code that used an instruction set that I had defined.

A key part of this section is looking at how I designed and implemented an intermediate representation of Goal in the form of a data structure. This intermediate representation was then simpler to compile into my low level instruction set. The importance of a good intermediate representation can be seen in \cite[p.~221]{EngComp2012} where they state;

\begin{displayquote}
Most passes in the compiler read and manipulate the IR form of code. Thus, decisions about what to represent and how to represent it play a  crucial role in both the cost of compilation and it's effectiveness. 
\end{displayquote}

In this section I will go into detail about the design of the data structure that you are required to parse down to, and which represents all of the features I implemented in my language. 

Although in this section I will be talking a lot about code generated using the instruction set defined in my executer, I will not be going into detail about the design of the executor or the low level language I created. For more information on this you can go to the next chapter that does focus more on the design and execution of the low level instruction set.


\section{Intermediate Representation}

Having a good intermediate representation of the program being compiled is very important part of creating a compiler. It is important to make sure that you do not misinterpret the program you are compiling and a good way to do this is to create a data structure that clearly represents what your program is doing, whilst ensuring this data structure is easy enough to compile into your target language.

\subsection{Introduction to Intermediate Representations}

An intermediate representation of a program is where you create some data structure of your program that can be more easily interpreted and handled by your compiler. It is also important to ensure your IR\footnotemark[1] is designed in such a way that you capture the meaning of the program you wish to represent, and do not end up misrepresenting your program.

\footnotetext[1]{IR is shorthand for Intermediate Representation}

There are several approaches to generating an IR. These can been seen in \cite[p.~223]{EngComp2012} where they state there are three main approaches to generating an IR; 

\begin{itemize}
\item Graphical IRs, This uses tree or graph data structures.  
\item Linear IRS, This will closely resemble pseudo-code. 
\item Hybrid IRS, A combination of both Linear and Graphical approaches.
\end{itemize}

I chose to go with a Hybrid IR. A most suitable approach for me because, as stated in \cite[p.~113]{CompDes2005}, Linear IR are often used when compiling for stack based virtual machines, which is the architecture of the virtual machine that I created. Also I wanted to take advantage of how easily Haskell allows you to create recursive data structures, that can be used when generating trees you would find in Graphical IRs. Hence why I decided to use a Hybrid IR, taking advantage of the positive aspects from both approaches .  

This approach provided me with a relatively high level representation of programs, meaning it gave a representation that would closely resemble the pseudo-code of the program you are trying to compile. This makes it nice format to work with as it makes it easier to visualize what is actually occurring during the later stages of code generation. 

It is important to note there are several different approaches I could have used when creating an intermediate representation and the choice to use a high level approach was both a design choice and also to do with ease of implementation. 

To best understand the approach I used to create the data structure for my intermediate representation it is good to follow a simple example of how you take a specific command then translate that into my intermediate representation.

\subsection{Example Creating an Intermediate Representation}

I will now show an example of the process I went through when generating my IR. The example I will use is creating a data structure that best represents an if statement. This is what an if statement looks like in Goal;

\begin{lstlisting}
	if (x < 1) {
		return x;
	};
\end{lstlisting}

You can see that there are two main parts to this statement. The condition directly after the if command,  $(x < 1)$  and then the code ,in this case $return$ $x;$, inside of the curly brackets. 

What is important to notice is that you could replace the return statement with any other acceptable code, even another If statement. Whereas the expression after the if is limited only to be certain things. In this case we could rewrite a general definition for an if statement to look like this;

\begin{lstlisting}
	if EXPRESSION {
		PROGRAM 
	}
\end{lstlisting}

This can be described by saying; If some expression is true, then execute the program inside the curly brackets.
 
Now all we need to do is define what  EXPRESSION and PROGRAM can be. Then we can create a data type in Haskell code to represent If statements that looks like this;

\begin{lstlisting}
	data IfStatement = If Expr Prog 
\end{lstlisting}

A program is quite easy to define, it's just a one or more instructions written to perform a specific task\footnotemark[2]. Therefore we must define all of the instructions that can be used to make a program in our data type Prog. Looking at our example that needs to include return statements, but I also said we were allowed to have nested if statements. Therefore we can create a recursive data structure that will allow this;

\footnotetext[2]{\url{wikipedia.org/wiki/Computer_program}}

\begin{lstlisting}
	data Prog = If Expr Prog | Return Expr
\end{lstlisting}


An expression is slightly harder to define. In the case of If statements we know our expression needs to do something. It needs to give us some condition which we can then decide is true or false. 

We can now create a data type for our expressions. We will need to allow for nested expressions but also allow for different comparison operators such as equals or less than. This can be shown here;

\begin{lstlisting}
	data Expr = ExprComp Op Expr Expr 
                  	| Val Number 
                 	| Var Name
 
	data Op	  = GET | LET | NEQ | EQU
\end{lstlisting}

Although this example gives us quite a strict definition of what an expression needs to be for an If statement later on we will need to come up with a more general definition that is more applicable to other instances of using expressions (such as in arithmetic operations). The language specification for Go describes an expression by saying; "An expression specifies the computation of a value by applying operators and functions to operators". Another way to think of an expression is it is any valid unit of code that resolves to a single value.


Now that we have our definitions for what Expressions and Programs can be, we can begin to group what instructions belong in which section. With all our expressions being defined in the data type Expr and all our instructions used to make programs in the data type Prog.

Therefore if we are only considering our If statement example from earlier (ignoring how we defined the variable  x) we end up with the following data structure;

\newpage

\begin{lstlisting}
	data Prog   = If Expr Prog | Return Expr

	data Expr   = ExprComp Op Expr Expr 
	               |Val Number 
	               |Var Name 

	data Op     = GET | LET | NEQ | EQU

	type Name   = Char

	type Number = Int
\end{lstlisting}

The above data structure now enables you to create high level intermediate representation of the original If statement we looked at, through the use of Haskell data types. Meaning our initial example can be represented by the following expression;

\begin{lstlisting}
   If (ExprComp LET (Var 'x') (Val 1)) (Return (Var 'x'))
\end{lstlisting} 

You can see here how the data structure has a broad scope actually allowing expressions such as $(4<6) != (4 < (5>6))$, which don't really make sense. But I felt it was better to allow for these bizarre expressions, and handle any stricter rules in parsing stages. Rather than create an IR that was too restrictive.

\subsection{Analysis and Expansion of Creating an IR Example}

The key point of the above example is to understand that each part of a languages features can be, and needs to be, strictly defined. I found it important to remember the data structure I am creating  has a purpose other than just being a new representation of Goal. It needs to actually extract the important pieces of information from the code, such that you are left with a series of concise statements that hold all the information necessary for you to start to rebuild the program using a new instruction set.

The example also highlights the process of abstracting out each part of a languages functionality, then choosing the appropriate structure to use to represent them. An important part of this example is showing the use of recursive data structures to generate tree like data types.

For example look at the code;

\begin{lstlisting}
	if (x > 1){
		if (x > 5){
		 	if (x == 6) {
		 		return 1;
		 	};	
		};
	};
\end{lstlisting}  
 
It will be represented using data structure from our example by the following Haskell expression;

\begin{lstlisting}
If (ExprComp GET (Var 'x') (Val 1)) 
	(If (ExprComp GET (Var 'x') (Val 5))
		(If (ExprComp EQU (Var 'x') (Val 6))
			(Return (Val 1)))) 
\end{lstlisting} 

This can also be visualized in a tree structure with branches to the right being the outcome if the condition is true and branches to the left if the condition is false;

\begin{figure}[h]
\centering
\begin{forest}
for tree={
  draw,
  minimum width=3cm, 
  minimum height=1cm,
  anchor=north,
  align=center,
  child anchor=north
},
[{$x > 1$}, align=center
  [{$break;$}]
  [{$x > 5$}
    [{$break;$}]
    [{$x == 6$}
  		[$break$]
  		[$return$ $1;$] 
  	]
  ]
]
\end{forest}
\caption{Visual representation of tree structure of nested if statements}
\label{fig:ifTree} 
\end{figure}

What figure~\ref{fig:ifTree} shows you is the importance of using recursive data structures when allowing an arbitrary number of nested commands. You will see that as you start to expand your data structure to include multiple function declarations, or even large conditional expressions, building a tree like data structure is a good way to represent any program.

Overall this example helps to introduce all the key concepts behind the creation of the IR I used in my compiler.

\subsection{My Intermediate Representation of Goal}

Below I show the data structure I created to as an intermediate representation of Goal;

%TODO check is up to date % 
\begin{lstlisting}
 data Prog        = CreateChan Name
                     | PushToChan Name Expr
                     | Assign Name Expr
                     | Show Expr
                     | Print String
                     | If Expr Prog
                     | IfElse Expr Prog Prog
                     | ElseIf Expr Prog [ElseIfCase'] Prog
                     | While Expr Prog
                     | Seqn [Prog]
                     | Empty                                     
                     | Return (Maybe Expr)                       
                     | Func FName [Name] Prog                     
                     | Main Prog                                 
                     | VoidFuncCall Name [Expr]
                     | GoCall Name [Expr]                                                                    
                     | Wait 
                     | Kill

 data ElseIfCase' = Case Expr Prog                                 

\end{lstlisting}

You can see that for almost every command you can use in Goal it has been broken down the same way that was shown in my example and represented in a psuedocode style whilst allowing the creation of tree like structures through the use of the recursive data structure.

One of the most important elements of my IR is the $Seqn [Prog]$ command. This is one of the key aspects of the IR and what allows for multiple commands within functions and even for allowing multiple functions.


%TODO add sources
\section{Code Generation}

Code generation is one of the most important aspects of compiling. It is the process of translating an intermediate representation of a program into code, without losing that programs meaning. In my case it meant creating functions that could translating my IR into code made up of the instruction set my executor could handle.  


\subsection{Brief Introduction to Target Language Instruction Set}

To understand how code generation works it is good to get a brief understanding of how my instruction set is structured and behaves. This section is just a brief introduction to the concepts that are necessary to understand in order for me to explain the process of generating code, more information on executing the code and the design of the instruction set used in my executor can be found in chapter 5.

My instruction set was designed so that you can translate any intermediate representation, no matter how complicated, into a list of instructions that can then be executed one by one. 

There are two important things to understand about my virtual machine, the platform which executes the code we are generating. First, is that it uses a stack, so any operations we wish to perform must do so using a stack. Secondly code execution is controlled by a program counter which starts at 0 and increments by one after every instruction, unless an instruction wants to jump to a another section of code then the program counter will be set based on where in the code you want to go. 

Don't worry if you are not to familiar with these concepts as they are gone over in more detail in the next chapter.    

 
\subsection{Example Code Generation}

I will now show an example of how you would generate code for an instruction from our data set. We will carry on with our earlier example of looking at an if statement now we know what the IR for this instruction is we should look at the code we wish to generate. 

\begin{lstlisting}
if(x < 1){
	return x;
};
\end{lstlisting}
Will become;

\begin{lstlisting}
If (CompExpr LET (Var x) (Val (Integer 1))) (Return (Var x))
\end{lstlisting}
	
Now I will show the code we need to generate;

\begin{lstlisting}
	[PUSHV "x",         
	 PUSH (Integer 1), 
	 COMP LET,         
	 JUMPZ i,           
	 PUSHV "x",
	 RSTOP,           
	 LABEL i]
	
\end{lstlisting}

To better understand this code you may want to refer quickly to chapter 5, but to give a brief understanding the variable $x$ is compared with $1$ then if this condition is true 1 is placed onto the stack, if not 0 is. Then $JUMPZ$ $i$ checks the top of the stack to see if it finds a 0, if so it will jump too $LABEL$ $i$ if not it moves on to return $x$. 

To generate code we will need to use pattern matching. If we first generate the type of our compiler function $comp$ we can see more easily exactly what it needs to do. 

\begin{lstlisting}
		comp  :: Prog -> Code
\end{lstlisting} 

Where $Code$ is the type that represents a list of instructions. If we look at our if statement we know that it is split nicely into two parts; the $Expr$ and the $Prog$. Therefore we can use this in our code generator. 

If we first focus on creating a function which can generate code for our expression.

\begin{lstlisting}
compE                  :: Expr -> Code
compE (CompExr o e1 e2)= compE x ++ compE y ++[COMP o]
compE (Val n)          = [PUSH n]
compE (Var v)          = [PUSHV v]
\end{lstlisting}  

This may seem like a bit of a jump but it's pretty simple to see whats going on here. It's good to think of the $Expr$ type as things that leave a result on the stack, and then our $Prog$ type as commands that do things based on whats on top of the stack.

Making $compE$ recursive is also useful in case we want to include more types of expressions, for example it allows us to easily expand this function to handle expressions such as $(5 + 6) == (12 - 1)$. We can make sure that any restrictions on expressions are dealt with in our parser. 

Now we need to look at how we can incorporate this function into solving our initial problem, generating code for if and return statements. Lets first try this; 

\begin{lstlisting}
	compP            :: Prog -> Code 
	compP (If e p1)  = ifDealer e p1  
	compP (Return e) = compE e ++ [RSTOP] 
\end{lstlisting}

Return is pretty simple to understand as it's just placing something on the stack and then using the call $RSTOP$. We now need to define our $ifDealer$ which creates the code for if statements. Remembering that we still want to allow all the features we mentioned in our earlier example.

\begin{lstlisting}
ifDealer     :: Expr -> Prog -> Code 
ifDealer e p = 
	compE e ++ [JUMPZ 1] ++ compP p ++ [LABEL 1]
\end{lstlisting}

This looks suitable but there are some issues, what if we have nested if statements, we would be using the same label everywhere so how would we know where to jump to? We need a way to generate a unique label for each time we want to use the $LABEL$ command.

\begin{figure}[h]
\centering
\begin{lstlisting}
    type State 	   =  Label

    data ST a 	   =  S (State -> (a, State))

    apply          :: ST a -> State -> (a,State)
    apply (S f) x  =  f x

 instance Monad ST where
    return x       =  S (\s -> (x,s))
    st >>= f   	   = S (\s -> let (x,s') = 
                        apply st s in apply (f x) s')

    fresh          :: ST Label
    fresh          = S (\s -> (s, s+1))

\end{lstlisting}
\caption{Definition of the state monad}
\label{fig:stMonad}
\end{figure}  

This is where we use the state monad. It can be used to generate a new label whenever we want and helps us make sure that we never have duplicated labels. In figure \ref{fig:stMonad} you can see the definition of the state monad and the function $fresh$ which is used every time we want to generate a new label. 

We now need to update our $compP$ and $ifDealer$ functions to make use of the state monad;

\begin{lstlisting}
compP            :: Prog -> ST Code 
compP (If e p1)  = ifDealer e p1  
compP (Return e) = return (compE e ++ [RSTOP])
	
ifDealer     :: Expr -> Prog -> Code 
ifDealer e p = 
	do i  <- fresh
	   pc <- compP p
	   return (compE e ++[JUMPZ i]++ pc ++[LABEL i])
\end{lstlisting}

This now will allow us to generate code for an arbitrary number of nested if statements. Though we now have the issue of that we now return $ST$ $Code$, when we need to be passing $Code$ to our virtual machine. 

Therefore we need to create another function to get rid of this for us. Which still has the type $Prog$ $->$ $Code$ that we want. We can do this by making use of the $apply$ function seen in figure \ref{fig:stMonad}.

\begin{lstlisting}
	comp    :: Prog -> Code
	comp p  =  fst (apply (compP p) (0))
\end{lstlisting}    

Now we can generate code in a format that can be run by our virtual machine.

\subsection[Analysis and Expansion of Code Generation Example]{Analysis and Expansion of \\ Code Generation Example}

The example is very useful in showing one of the most important features of my code generation section, the use of the state monad to ensure that the creation of each label is unique and allows for looping and conditional expressions in the language I am compiling.

It also shows the importance of using pattern matching and recursive functions to be able to efficiently generate the correct output and also allow for an arbitrary number of nested expressions with the code being compiled. It also helps to highlight why Haskell is such a useful language to use in compilers more specifically in projects where code generation is the major focus.

Overall this example highlights very nicely exactly how my code generator works. In very similar fashion to the designing of my IR I wanted to make sure I could handle all commands exactly as needed through the use of pattern matching. Then, use recursion, to allow for some arbitrary number of nested commands. Finally I used monads to ensure the creation of unique labels and tidy up my implementation.

There is one improvement that could be made to my implementation of my code generator. Including a writer monad and making use of monad transformers to still include the state monad would tidy up my implementation. Getting rid of messy long list concatenations.  


